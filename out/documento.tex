\documentclass[12pt,spanish,]{article}
\usepackage{lmodern}
\usepackage{amssymb,amsmath}
\usepackage{ifxetex,ifluatex}
\usepackage{fixltx2e} % provides \textsubscript
\ifnum 0\ifxetex 1\fi\ifluatex 1\fi=0 % if pdftex
  \usepackage[T1]{fontenc}
  \usepackage[utf8]{inputenc}
\else % if luatex or xelatex
  \ifxetex
    \usepackage{mathspec}
  \else
    \usepackage{fontspec}
  \fi
  \defaultfontfeatures{Ligatures=TeX,Scale=MatchLowercase}
    \setmainfont[]{Ubuntu}
    \setmonofont[Mapping=tex-ansi]{Ubuntu Mono}
\fi
% use upquote if available, for straight quotes in verbatim environments
\IfFileExists{upquote.sty}{\usepackage{upquote}}{}
% use microtype if available
\IfFileExists{microtype.sty}{%
\usepackage{microtype}
\UseMicrotypeSet[protrusion]{basicmath} % disable protrusion for tt fonts
}{}
\usepackage[a4paper]{geometry}
\usepackage{hyperref}
\PassOptionsToPackage{usenames,dvipsnames}{color} % color is loaded by hyperref
\hypersetup{unicode=true,
            pdftitle={Usando Pandoc},
            pdfauthor={Sergio Alvariño salvari@gmail.com},
            colorlinks=true,
            linkcolor=Maroon,
            citecolor=Blue,
            urlcolor=Blue,
            breaklinks=true}
\urlstyle{same}  % don't use monospace font for urls
\ifnum 0\ifxetex 1\fi\ifluatex 1\fi=0 % if pdftex
  \usepackage[shorthands=off,main=spanish]{babel}
\else
  \usepackage{polyglossia}
  \setmainlanguage[]{spanish}
\fi
\IfFileExists{parskip.sty}{%
\usepackage{parskip}
}{% else
\setlength{\parindent}{0pt}
\setlength{\parskip}{6pt plus 2pt minus 1pt}
}
\setlength{\emergencystretch}{3em}  % prevent overfull lines
\providecommand{\tightlist}{%
  \setlength{\itemsep}{0pt}\setlength{\parskip}{0pt}}
\setcounter{secnumdepth}{5}
% Redefines (sub)paragraphs to behave more like sections
\ifx\paragraph\undefined\else
\let\oldparagraph\paragraph
\renewcommand{\paragraph}[1]{\oldparagraph{#1}\mbox{}}
\fi
\ifx\subparagraph\undefined\else
\let\oldsubparagraph\subparagraph
\renewcommand{\subparagraph}[1]{\oldsubparagraph{#1}\mbox{}}
\fi

\title{Usando Pandoc}
\author{Sergio Alvariño
\href{mailto:salvari@gmail.com}{\nolinkurl{salvari@gmail.com}}}
\date{abril-2016}

\begin{document}
\maketitle
\begin{abstract}
Una chuleta para usar Pandoc, copiado descaradamente de varios sitios de
internet.

Solo para referencia rápida y personal.
\end{abstract}

{
\hypersetup{linkcolor=black}
\setcounter{tocdepth}{3}
\tableofcontents
}
\section{¿Cómo usar esto?}\label{cuxf3mo-usar-esto}

\subsection{Muy rápido}\label{muy-ruxe1pido}

Clona el repo en un directorio :

\begin{verbatim}
git clone https://bitbucket.org/salvari/pandoc_basico
\end{verbatim}

Renombra el directorio:

\begin{verbatim}
mv pandoc_basico miProyecto
\end{verbatim}

Elimina la info de git

\begin{verbatim}
rm -rf miProyecto/.git
\end{verbatim}

Edita el fichero miProyecto/src/documento.md con tu editor de texto
favorito.

Ejecuta:

\begin{description}
\item[make]
Para generar todos los ficheros de salida y el fichero README.md
(equivale a \emph{make all})
\item[make clean]
Para borrar todos los ficheros de salida
\item[make reset]
Equivale a \emph{make clean all}
\end{description}

\subsection{Más detalles}\label{muxe1s-detalles}

El makefile está preparado para procesar \textbf{todos} los ficheros con
extensión \emph{.md} que haya en el directorio \emph{src}. Esto permite
escribir documentos largos y dividirlos en secciones, por ejemplo
podríamos tener los siguientes documentos en el directorio \emph{src}

\begin{verbatim}
00_Comienzo.md
10_Capitulo_01.md
20_Capitulo_02.md
30_Conclusion.md
40_apendices.md
\end{verbatim}

Al ejecutar make nos crearía \textbf{un solo documento de salida}
concatenando todos los ficheros. El orden en que los concatena es el
orden en el que aparecen al hacer un \emph{ls} por eso se nombran con
una numeración al principio que permita ordenarlos a gusto del autor.

Si quieres cambiar el nombre del fichero de salida (\emph{documento})
tendrás que editar el makefile y cambiar la línea:

\begin{verbatim}
target  := documento
\end{verbatim}

Otras líneas que puedes tocar en el makefile son las que especifican el
idioma y los tipos de letra usados.

\section{¿Qué es Pandoc?}\label{quuxe9-es-pandoc}

Como explican en http://pandoc.org, Pandoc es una librería en Haskell
para hacer conversión de documentos de un formato markup a otro. Y
también es una herramienta de terminal de comandos que usa esa librería.

Lo que nos permite Pandoc a la hora de documentar un proyecto es
mantener la documentación en un formato abierto y sencillo (markdown) y
generar salidas en distintos formatos (pdf, mediawiki, epub, html, etc)
con un simple comando.

\section{¿Qué necesitas tener
instalado?}\label{quuxe9-necesitas-tener-instalado}

\begin{itemize}
\tightlist
\item
  Pandoc
\item
  make
\item
  git (no es imprescindible pero muy recomendable)
\item
  Las plantillas de Pandoc (o \emph{templates})
\item
  Un buen editor de texto
\end{itemize}

\section{Chuletario de Pandoc}\label{chuletario-de-pandoc}

\subsection{Backslash Escapes}\label{backslash-escapes}

Salvo que estemos dentro de un bloque de código o de ``código en
linea'', \textbf{cualquier carácter de puntuación o espacio} precedido
de contrabarra se tratará de forma literal, incluso si ese carácter
normalmente indique algún formato.

\subsection{Bloque de título}\label{bloque-de-tuxedtulo}

Es una forma rápida de indicar el título el autor o autores y la fecha.
Tiene que ir al principio del documento

\begin{verbatim}
% título
% autor(es) (separados por :)
% fecha
\end{verbatim}

Alternativamente se puede usar otro estilo para el bloque de título,
mucho más completo, en formato
\href{https://en.wikipedia.org/wiki/YAML}{YAML}, especificando
variables. No puede usarse simultáneamente con el anterior, hay que
escoger entre los dos estilos.

Se pueden especificar todo tipo de variables \footnote{Ojo por que en el
  makefile propuesto se especifica el lenguaje, asi que la variable del
  bloque de título no va a tener efecto en este caso.}.

\begin{verbatim}
---
title: Título
author:
- Autor Uno <autor.uno@correo.com>
- Otro autor <otroautor@correo.com>
tags: [nothing, nothingness]
date: enero-2016
lang: es-ES
abstract: |
  Este es el resumen.

  Con dos párrafos.
...
---
\end{verbatim}

\subsection{Incrustar TeX y HTML}\label{incrustar-tex-y-html}

\begin{itemize}
\tightlist
\item
  Los comandos TeX se pasan de forma transparente al Markdown, y afectan
  solo a la salida de LaTeX y ConTeXt; en el resto de casos se borran
\item
  El código HTML pasará a la salida sin cambios, pero el Markdown dentro
  de los bloques HTML se procesa como Markdown
\end{itemize}

\subsection{Párrafos y retornos de
línea}\label{puxe1rrafos-y-retornos-de-luxednea}

\begin{itemize}
\tightlist
\item
  Un párrafo es una o más líneas de texto separadas por una linea en
  blanco del resto
\item
  Una línea que termina con dos espacios, o una línea que termina con un
  fin de linea escapado (contrabarra seguida de retorno de linea) indica
  un cambio de linea manual
\end{itemize}

\subsection{Itálica, negrita, superescrito, subesctrito,
tachado}\label{ituxe1lica-negrita-superescrito-subesctrito-tachado}

\begin{verbatim}
*Itálica* and **negrita** se indican con asteriscos.

Para  ~~tachar~~ texto usa tildes dobles.

Superscrito se indica así: 2^ndo^.

Subescrito con tildes simples, así: H~2~O.

Los espacios en el superescrito y el subescrito tienen que ir escapados,
p.ej., H~esto\ es \ un\ subescrito~.
\end{verbatim}

\subsection{TeX matématico o código incrustado en
linea}\label{tex-matuxe9matico-o-cuxf3digo-incrustado-en-linea}

\begin{verbatim}
El TeX matemático va entre signos$: $2 + 2$.

El código en linea va entre comillas invertidas: `echo 'hello'`
\end{verbatim}

\subsection{Enlaces e imágenes}\label{enlaces-e-imuxe1genes}

\begin{verbatim}
<http://example.com>
<foo@bar.com>
[inline link](http://example.com "Title")
![inline image](/path/to/image, "alt text")

[reference link][id]
[implicit reference link][]
![reference image][id2]

[id]: http://example.com "Title"
[implicit reference link]: http://example.com
[id2]: /path/to/image "alt text"
\end{verbatim}

\subsection{Notas al pie de página}\label{notas-al-pie-de-puxe1gina}

\begin{verbatim}
Las notas en linea son como
esta.^[Nótese que las notas en linea no pueden tener más de un párrafo.]
Las notas de referencia son como esta.[^id]

[^id]:  Las notas de referencia pueden contener varios párrafos.

    Los parámetros a continuación deben estar identados.
\end{verbatim}

\subsection{Citas}\label{citas}

\begin{verbatim}
Blah blah [see @doe99, pp. 33-35; also @smith04, ch. 1].

Blah blah [@doe99, pp. 33-35, 38-39 and *passim*].

Blah blah [@smith04; @doe99].

Smith says blah [-@smith04].

@smith04 says blah.

@smith04 [p. 33] says blah.
\end{verbatim}

\subsection{Encabezados}\label{encabezados}

\begin{verbatim}
Encabezado 1
========

Encabezado 2
--------

# Encabezado 1 #

## Encabezado 2 ##
\end{verbatim}

Las almohadillas de cierre \# son opcionales. Es necesario añadir una
línea en blanco antes y después de cada cabecera.

\subsection{Listas}\label{listas}

\paragraph{Listas Ordenadas}\label{listas-ordenadas}

\begin{verbatim}
1. example
2. example

A) example
B) example
\end{verbatim}

\paragraph{Listas desordenadas}\label{listas-desordenadas}

Los items de la lista deben ir marcados con `*', `+', or `-'.

\begin{verbatim}
+   example
-   example
*   example
\end{verbatim}

Las listas se pueden anidar de la forma usual:

\begin{verbatim}
+   example
    +   example
+   example
\end{verbatim}

\paragraph{Listas de definición}\label{listas-de-definiciuxf3n}

\begin{verbatim}
Term 1

:   Definition 1

Term 2

:   Definition 2
    Second paragraph of definition 2.
\end{verbatim}

\subsection{Blockquotes}\label{blockquotes}

\begin{verbatim}
>   blockquote
>>  nested blockquote
\end{verbatim}

Es necesario añadir lineas en blanco antes y después de los
bloques-cita.

\subsection{Tablas}\label{tablas}

\begin{verbatim}
  Right     Left     Center     Default
-------     ------ ----------   -------
     12     12        12            12
    123     123       123          123
      1     1          1             1

Table:  Demonstration of simple table syntax.
\end{verbatim}

(Para tablas más complejas consulta la
\href{http://pandoc.org/README.html\#tables}{documentación de Pandoc}.)

\subsection{Bloques de código}\label{bloques-de-cuxf3digo}

Los bloques de código empiezan con tres o más tildes; y acaban por lo
menos con el mismo número de tildes:

\begin{verbatim}
~~~~~~~
{code here}
~~~~~~~
\end{verbatim}

Opcionalmente, se puede especificar el lenguaje que corresponde al
bloque de código:

\begin{verbatim}
~~~~~~~~~~~~~~~~~~~~~~~~~~~~~~~~~~~~~~~~~~ {.haskell .numberLines}
qsort []     = []
qsort (x:xs) = qsort (filter (< x) xs) ++ [x] ++
               qsort (filter (>= x) xs)
~~~~~~~~~~~~~~~~~~~~~~~~~~~~~~~~~~~~~~~~~~~~~~~~~~~~
\end{verbatim}

\subsection{Lineas horizontales}\label{lineas-horizontales}

3 o mas guiones o asteriscos en una linea (se permiten espacios
intercalados)

\begin{verbatim}
---
* * *
- - - -
\end{verbatim}

\subsection{Bloques verbatim}\label{bloques-verbatim}

Todo el texto identado cuatro espacios

\begin{verbatim}
Ejemplo Esto es un bloque verbatim y por ejemplo *esto* aparece
tal cual y no en itálica.
\end{verbatim}

\section{En que me he basado (o copiado si lo
prefieres)}\label{en-que-me-he-basado-o-copiado-si-lo-prefieres}

\begin{itemize}
\tightlist
\item
  En la \href{http://pandoc.org/README.html}{guia de usuario de Pandoc}
  Importante leersela para sacarle todo el jugo a esta herramienta
\item
  En la
  \href{https://github.com/dsanson/Pandoc.tmbundle/blob/master/Support/doc/cheatsheet.markdown}{chuleta
  de Pandoc} de \href{https://github.com/dsanson}{David Sanson},
  perfecta para referencia rápida
\item
  Para hacer el makefile me he leido varios tutoriales y copiado
  descaradamente de varios sitios que olvidé apuntar (lo siento)
\end{itemize}

\end{document}
