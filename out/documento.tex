\documentclass[12pt,spanish,]{article}
\usepackage{lmodern}
\usepackage{amssymb,amsmath}
\usepackage{ifxetex,ifluatex}
\usepackage{fixltx2e} % provides \textsubscript
\ifnum 0\ifxetex 1\fi\ifluatex 1\fi=0 % if pdftex
  \usepackage[T1]{fontenc}
  \usepackage[utf8]{inputenc}
\else % if luatex or xelatex
  \ifxetex
    \usepackage{mathspec}
  \else
    \usepackage{fontspec}
  \fi
  \defaultfontfeatures{Ligatures=TeX,Scale=MatchLowercase}
    \setmainfont[]{Source Sans Pro}
    \setmonofont[Mapping=tex-ansi]{Source Code Pro}
\fi
% use upquote if available, for straight quotes in verbatim environments
\IfFileExists{upquote.sty}{\usepackage{upquote}}{}
% use microtype if available
\IfFileExists{microtype.sty}{%
\usepackage{microtype}
\UseMicrotypeSet[protrusion]{basicmath} % disable protrusion for tt fonts
}{}
\usepackage[a4paper]{geometry}
\usepackage{hyperref}
\hypersetup{unicode=true,
            pdftitle={Pandoc Cheatsheet},
            pdfauthor={Sergio Alvariño copiado descaradamente de David Sanson},
            pdfborder={0 0 0},
            breaklinks=true}
\urlstyle{same}  % don't use monospace font for urls
\ifnum 0\ifxetex 1\fi\ifluatex 1\fi=0 % if pdftex
  \usepackage[shorthands=off,main=spanish]{babel}
\else
  \usepackage{polyglossia}
  \setmainlanguage[]{spanish}
\fi
\IfFileExists{parskip.sty}{%
\usepackage{parskip}
}{% else
\setlength{\parindent}{0pt}
\setlength{\parskip}{6pt plus 2pt minus 1pt}
}
\setlength{\emergencystretch}{3em}  % prevent overfull lines
\providecommand{\tightlist}{%
  \setlength{\itemsep}{0pt}\setlength{\parskip}{0pt}}
\setcounter{secnumdepth}{5}
% Redefines (sub)paragraphs to behave more like sections
\ifx\paragraph\undefined\else
\let\oldparagraph\paragraph
\renewcommand{\paragraph}[1]{\oldparagraph{#1}\mbox{}}
\fi
\ifx\subparagraph\undefined\else
\let\oldsubparagraph\subparagraph
\renewcommand{\subparagraph}[1]{\oldsubparagraph{#1}\mbox{}}
\fi

\title{Pandoc Cheatsheet\footnote{Cobbled together from
  \url{http://daringfireball.net/projects/markdown/syntax} and
  \url{http://johnmacfarlane.net/pandoc/README.html}.}}
\author{Sergio Alvariño copiado descaradamente de David Sanson}
\date{Abril 2016}

\begin{document}
\maketitle

{
\setcounter{tocdepth}{3}
\tableofcontents
}
Solo para referencia rápida y personal.

\section{Backslash Escapes}\label{backslash-escapes}

Salvo que estemos dentro de un bloque de código o de ``código en
linea'', \textbf{cualquier carácter de puntuación o espacio} precedido
de contrabarra se tratará de forma literal, incluso si ese carácter
normalmente indique algún formato.

\section{Bloque de título}\label{bloque-de-tuxedtulo}

Es una forma rápida de indicar el título el autor o autores y la fecha.

\begin{verbatim}
% título
% autor(es) (separados por :)
% fecha
\end{verbatim}

Un bloque de título mucho más completo:

\begin{verbatim}
---
title: Título
author:
- Autor Uno <autor.uno@gmail.com>
- Otro autor <otroautor@gmail.com>
tags: [nothing, nothingness]
date: enero-2016
lang: es-ES
abstract: |
  Este es el resumen.

  Con dos párrafos.
...
\end{verbatim}

\section{Incrustar TeX y HTML}\label{incrustar-tex-y-html}

\begin{itemize}
\tightlist
\item
  Los comandos TeX se pasan de forma transparente al Markdown, y afectan
  solo a la salida de LaTeX y ConTeXt; en el resto de casos se borran
\item
  El código HTML pasará a la salida sin cambios, pero el Markdown dentro
  de los bloques HTML se procesa como Markdown
\end{itemize}

\section{Párrafos y retornos de
línea}\label{puxe1rrafos-y-retornos-de-luxednea}

\begin{itemize}
\tightlist
\item
  Un párrafo es una o más líneas de texto separadas por una linea en
  blanco del resto
\item
  Una línea que termina con dos espacios, o una línea que termina con un
  fin de linea escapado (contrabarra seguida de retorno de linea) indica
  un cambio de linea manual
\end{itemize}

\section{Itálica, negrita, superescrito, subesctrito,
tachado}\label{ituxe1lica-negrita-superescrito-subesctrito-tachado}

\begin{verbatim}
*Itálica* and **negrita** se indican con asteriscos.

Para  ~~tachar~~ texto usa tildes dobles.

Superscrito se indica así: 2^ndo^.

Subescrito con tildes simples, así: H~2~O.

Los espacios en el superescrito y el subescrito tienen que ir escapados,
p.ej., H~esto\ es \ un\ subescrito~.
\end{verbatim}

\section{TeX matématico o código incrustado en
linea}\label{tex-matuxe9matico-o-cuxf3digo-incrustado-en-linea}

\begin{verbatim}
El TeX matemático va entre signos$: $2 + 2$.

El código en linea va entre comillas invertidas: `echo 'hello'`
\end{verbatim}

\section{Enlaces e imágenes}\label{enlaces-e-imuxe1genes}

\begin{verbatim}
<http://example.com>
<foo@bar.com>
[inline link](http://example.com "Title")
![inline image](/path/to/image, "alt text")

[reference link][id]
[implicit reference link][]
![reference image][id2]

[id]: http://example.com "Title"
[implicit reference link]: http://example.com
[id2]: /path/to/image "alt text"
\end{verbatim}

\section{Notas al pie de página}\label{notas-al-pie-de-puxe1gina}

\begin{verbatim}
Las notas en linea son como esta.^[Nótese que las notas en linea no pueden tener más de un párrafo.] Las notas de referencia son como esta.[^id]

[^id]:  Las notas de referencia pueden contener varios párrafos.

    Los parámetros a continuación deben estar identados.
\end{verbatim}

\section{Citas}\label{citas}

\begin{verbatim}
Blah blah [see @doe99, pp. 33-35; also @smith04, ch. 1].

Blah blah [@doe99, pp. 33-35, 38-39 and *passim*].

Blah blah [@smith04; @doe99].

Smith says blah [-@smith04].

@smith04 says blah.

@smith04 [p. 33] says blah.
\end{verbatim}

\section{Encabezados}\label{encabezados}

\begin{verbatim}
Encabezado 1
========

Encabezado 2
--------

# Encabezado 1 #

## Encabezado 2 ##
\end{verbatim}

Las almohadillas de cierre \# son opcionales. Es necesario añadir una
línea en blanco antes y después de cada cabecera.

\section{Listas}\label{listas}

\subsection{Listas Ordenadas}\label{listas-ordenadas}

\begin{verbatim}
1. example
2. example

A) example
B) example
\end{verbatim}

\subsection{Lisas desordenadas}\label{lisas-desordenadas}

Los items de la lista deben ir marcados con `*', `+', or `-'.

\begin{verbatim}
+   example
-   example
*   example
\end{verbatim}

Las listas se pueden anidar de la forma usual:

\begin{verbatim}
+   example
    +   example
+   example
\end{verbatim}

\subsection{Listas de definición}\label{listas-de-definiciuxf3n}

\begin{verbatim}
Term 1
  ~ Definition 1
Term 2
  ~ Definition 2a
  ~ Definition 2b

Term 1
:   Definition 1
Term 2
:   Definition 2
    Second paragraph of definition 2.
\end{verbatim}

\section{Blockquotes}\label{blockquotes}

\begin{verbatim}
>   blockquote
>>  nested blockquote
\end{verbatim}

Es necesario añadir lineas en blanco antes y después de los
bloques-cita.

\section{Tablas}\label{tablas}

\begin{verbatim}
  Right     Left     Center     Default
-------     ------ ----------   -------
     12     12        12            12
    123     123       123          123
      1     1          1             1

Table:  Demonstration of simple table syntax.
\end{verbatim}

(Para tablas más complejas consulta la documentación de Pandoc.)

\section{Bloques de código}\label{bloques-de-cuxf3digo}

Los bloques de código empiezan con tres o más tildes; y acaban por lo
menos con el mismo número de tildes:

\begin{verbatim}
~~~~~~~
{code here}
~~~~~~~
\end{verbatim}

Opcionalmente, se puede especificar el lenguaje que corresponde al
bloque de código:

\begin{verbatim}
~~~~~~~~~~~~~~~~~~~~~~~~~~~~~~~~~~~~~~~~~~ {.haskell .numberLines}
qsort []     = []
qsort (x:xs) = qsort (filter (< x) xs) ++ [x] ++
               qsort (filter (>= x) xs)
~~~~~~~~~~~~~~~~~~~~~~~~~~~~~~~~~~~~~~~~~~~~~~~~~~~~
\end{verbatim}

\section{Lineas horizontales}\label{lineas-horizontales}

3 o mas guiones o asteriscos en una linea (se permiten espacios
intercalados)

\begin{verbatim}
---
* * *
- - - -
\end{verbatim}

\section{Bloques verbatim}\label{bloques-verbatim}

Todo el texto identado cuatro espacios

\begin{verbatim}
Ejemplo Esto es un bloque verbatim y por ejemplo *esto* aparece
tal cual y no en itálica.
\end{verbatim}

\end{document}
